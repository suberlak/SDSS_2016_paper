% mnras_template.tex
%
% LaTeX template for creating an MNRAS paper
%
% v3.0 released 14 May 2015
% (version numbers match those of mnras.cls)
%
% Copyright (C) Royal Astronomical Society 2015
% Authors:
% Keith T. Smith (Royal Astronomical Society)

% Change log
%
% v3.0 May 2015
%    Renamed to match the new package name
%    Version number matches mnras.cls
%    A few minor tweaks to wording
% v1.0 September 2013
%    Beta testing only - never publicly released
%    First version: a simple (ish) template for creating an MNRAS paper

%%%%%%%%%%%%%%%%%%%%%%%%%%%%%%%%%%%%%%%%%%%%%%%%%%
% Basic setup. Most papers should leave these options alone.
\documentclass[fleqn,usenatbib]{mnras}  % a4paper,

% MNRAS is set in Times font. If you don't have this installed (most LaTeX
% installations will be fine) or prefer the old Computer Modern fonts, comment
% out the following line
%\usepackage{newtxtext,newtxmath}
%\usepackage{lmodern}
% Depending on your LaTeX fonts installation, you might get better results with one of these:
\usepackage{mathptmx}
%\usepackage{txfonts}


% Use vector fonts, so it zooms properly in on-screen viewing software
% Don't change these lines unless you know what you are doing
\usepackage[T1]{fontenc}
\usepackage{ae,aecompl}
\usepackage{slashbox}

%%%%% AUTHORS - PLACE YOUR OWN PACKAGES HERE %%%%%

% Only include extra packages if you really need them. Common packages are:
\usepackage{graphicx}	% Including figure files
\usepackage{amsmath}	% Advanced maths commands
\usepackage{amssymb}	% Extra maths symbols
\usepackage{savesym}  % prevent symbol conflicts
\savesymbol{sf}
%\generate{%
%  \file{breqn.sty}{\nopreamble\from{breqn.dtx}{breqn.sty}}%
%}
%\usepackage{breqn} % automatic breaking equation 
%\usepackage{fancyvrb}
%\VerbatimFootnotes
\usepackage{cprotect}  % to allow verb in caption 
\DeclareMathOperator\erfc{erfc}
\DeclareMathOperator\erf{erf}
\DeclareMathOperator\cdf{cdf}
\DeclareMathOperator\sf{sf}
\DeclareMathOperator\isf{isf}
\DeclareMathOperator\ppf{ppf}

%%%%%%%%%%%%%%%%%%%%%%%%%%%%%%%%%%%%%%%%%%%%%%%%%%

%%%%% AUTHORS - PLACE YOUR OWN COMMANDS HERE %%%%%

% Please keep new commands to a minimum, and use \newcommand not \def to avoid
% overwriting existing commands. Example:
%\newcommand{\pcm}{\,cm$^{-2}$}	% per cm-squared

%%%%%%%%%%%%%%%%%%%%%%%%%%%%%%%%%%%%%%%%%%%%%%%%%%

%%%%%%%%%%%%%%%%%%% TITLE PAGE %%%%%%%%%%%%%%%%%%%

% Title of the paper, and the short title which is used in the headers.
% Keep the title short and informative.
\title[SDSS Quasars]{SDSS Stripe 82 : quasar variability from forced photometry}

% The list of authors, and the short list which is used in the headers.
% If you need two or more lines of authors, add an extra line using \newauthor
\author[K. Suberlak et al.]{
Krzysztof Suberlak,$^{1}$\thanks{E-mail: suberlak@uw.edu}
\v{Z}eljko Ivezi\'c, $^{1}$
Yusra AlSayyad,$^{1}$ 
\\
% List of institutions
$^{1}$Department of Astronomy, University of Washington, Seattle, WA, United States\\
}

% These dates will be filled out by the publisher
\date{Accepted XXX. Received YYY; in original form ZZZ}

% Enter the current year, for the copyright statements etc.
\pubyear{2015}

% Don't change these lines
\begin{document}
\label{firstpage}
\pagerange{\pageref{firstpage}--\pageref{lastpage}}
\maketitle

% Abstract of the paper
\begin{abstract}

\end{abstract}

%%%%%%%%%%%%%%%%%%%%%%%%%%%%%%%%%%%%%%%%%%%%%%%%%%

%%%%%%%%%%%%%%%%% BODY OF PAPER %%%%%%%%%%%%%%%%%%


\subsection{Faint Sources}
 
Forced photometry measurement in a single epoch  is in fact the mean of a likelihood for the observed flux. We can think of each source as  being represented by a Gaussian likelihood centered on the measured flux $F_{i}$, with a widh corresponding to the measurement error $\sigma_{i}$. Thus in the flux likelihood space bright sources have very narrow Gaussians, with a width on the level of $1-2 \%  \approx 0.01-0.02$ mag, whereas faint sources with larger uncertainties have very wide Gaussian tails, that can extend  below zero.  Such nonzero likelihood for a negative flux measurement is unphysical, given that we know that no source can in  reality have a negative flux. Any negative portion of the likelihood stems from the background fluctuation, as described in previous section, or from very large measurement error. We address the issue of negative tails of Gaussian flux likelihood by recalculating the measurement of flux for all sources  with less than $'2\sigma'$ detection, i.e. $ \langle F_{L} \rangle  < k \sigma$, with $k=2$. This corresponds to the $2\%$ probability of $F_{L} < 0$  (in a  Gaussian likelihood).

%%%%%%%%%%%%%%%%%%%%%%%%%%%%%%%%%%%%%%%%%%%%%%%%%%

%%%%%%%%%%%%%%%%%%%% REFERENCES %%%%%%%%%%%%%%%%%%

% The best way to enter references is to use BibTeX:

%\bibliographystyle{mnras}
%\bibliography{example} % if your bibtex file is called example.bib

%%%%%%%%%%%%%%%%%%%%%%%%%%%%%%%%%%%%%%%%%%%%%%%%%%


% Don't change these lines
\bsp	% typesetting comment
\label{lastpage}
\end{document}

% End of mnras_template.tex
