% mnras_template.tex
%
% LaTeX template for creating an MNRAS paper
%
% v3.0 released 14 May 2015
% (version numbers match those of mnras.cls)
%
% Copyright (C) Royal Astronomical Society 2015
% Authors:
% Keith T. Smith (Royal Astronomical Society)

% Change log
%
% v3.0 May 2015
%    Renamed to match the new package name
%    Version number matches mnras.cls
%    A few minor tweaks to wording
% v1.0 September 2013
%    Beta testing only - never publicly released
%    First version: a simple (ish) template for creating an MNRAS paper

%%%%%%%%%%%%%%%%%%%%%%%%%%%%%%%%%%%%%%%%%%%%%%%%%%
% Basic setup. Most papers should leave these options alone.
\documentclass[fleqn,usenatbib]{mnras}  % a4paper,

% MNRAS is set in Times font. If you don't have this installed (most LaTeX
% installations will be fine) or prefer the old Computer Modern fonts, comment
% out the following line
%\usepackage{newtxtext,newtxmath}
%\usepackage{lmodern}
% Depending on your LaTeX fonts installation, you might get better results with one of these:
\usepackage{mathptmx}
%\usepackage{txfonts}


% Use vector fonts, so it zooms properly in on-screen viewing software
% Don't change these lines unless you know what you are doing
\usepackage[T1]{fontenc}
\usepackage{ae,aecompl}
\usepackage{slashbox}

%%%%% AUTHORS - PLACE YOUR OWN PACKAGES HERE %%%%%

% Only include extra packages if you really need them. Common packages are:
\usepackage{graphicx}	% Including figure files
\usepackage{amsmath}	% Advanced maths commands
\usepackage{amssymb}	% Extra maths symbols
\usepackage{savesym}  % prevent symbol conflicts
\savesymbol{sf}
%\generate{%
%  \file{breqn.sty}{\nopreamble\from{breqn.dtx}{breqn.sty}}%
%}
%\usepackage{breqn} % automatic breaking equation 
%\usepackage{fancyvrb}
%\VerbatimFootnotes
\usepackage{cprotect}  % to allow verb in caption 
\DeclareMathOperator\erfc{erfc}
\DeclareMathOperator\erf{erf}
\DeclareMathOperator\cdf{cdf}
\DeclareMathOperator\sf{sf}
\DeclareMathOperator\isf{isf}
\DeclareMathOperator\ppf{ppf}

%%%%%%%%%%%%%%%%%%%%%%%%%%%%%%%%%%%%%%%%%%%%%%%%%%

%%%%% AUTHORS - PLACE YOUR OWN COMMANDS HERE %%%%%

% Please keep new commands to a minimum, and use \newcommand not \def to avoid
% overwriting existing commands. Example:
%\newcommand{\pcm}{\,cm$^{-2}$}	% per cm-squared

%%%%%%%%%%%%%%%%%%%%%%%%%%%%%%%%%%%%%%%%%%%%%%%%%%

%%%%%%%%%%%%%%%%%%% TITLE PAGE %%%%%%%%%%%%%%%%%%%

% Title of the paper, and the short title which is used in the headers.
% Keep the title short and informative.
\title[SDSS Quasars]{SDSS Stripe 82 : quasar variability from forced photometry}

% The list of authors, and the short list which is used in the headers.
% If you need two or more lines of authors, add an extra line using \newauthor
\author[K. Suberlak et al.]{
Krzysztof Suberlak,$^{1}$\thanks{E-mail: suberlak@uw.edu}
\v{Z}eljko Ivezi\'c, $^{1}$
Yusra AlSayyad,$^{1}$ 
\\
% List of institutions
$^{1}$Department of Astronomy, University of Washington, Seattle, WA, United States\\
}

% These dates will be filled out by the publisher
\date{Accepted XXX. Received YYY; in original form ZZZ}

% Enter the current year, for the copyright statements etc.
\pubyear{2016}

% Don't change these lines
\begin{document}
\label{firstpage}
\pagerange{\pageref{firstpage}--\pageref{lastpage}}
\maketitle

% Abstract of the paper
\begin{abstract}

\end{abstract}

%%%%%%%%%%%%%%%%%%%%%%%%%%%%%%%%%%%%%%%%%%%%%%%%%%

%%%%%%%%%%%%%%%%% BODY OF PAPER %%%%%%%%%%%%%%%%%%


\section{Assignment 6 paragraph}
\subsection{before}
Each measurement of flux is affected by the background noise.  The bright optical background can have two contributions to faint source detection. First, the oscillation of background around the mean may lead to spurious detections. We can understand it by modelling the distribution of background counts as a Gaussian centered around the mean value $B_{0}$ :   $B-B_{0}  \sim  \mathcal{N}(0,\sigma_{B})$. The noise itself is Poissonian, and for a large number of photons hitting the detector the width of the distribution $\sigma_{B}$ is proportional to the square-root of counts: $\sqrt {B}$. Thus on a 4kx4k CCD, similar to those used by SDSS, we would expect about 1 false detection in a million at $5\sigma$ level - 16 per CCD.  
Second contribution of background variation is the unphysically low flux at source location. Since the background oscillates around a mean value, an individual epoch may have a lower-than-average background value which after mean background subtraction yields a negative value of flux at source location. Thus especially for variable sources, the location where it was detected in coadds may have a negative flux value because if the source is below detection threshold in an individual epoch, we are measuring the background noise oscillation. The background noise can significantly affect the measurement if it is as strong as the flux of a source in some epochs. 
\subsection{after}
Forced photometry in a background-subtracted epoch may yield unphysical, negative values. Such negative pixel value may originate from the variation of background across the image. If we model the background counts  as a Gaussian centered around the mean  $B_{0}$ :   $B-B_{0}  \sim  \mathcal{N}(0,\sigma_{B})$, then background in some parts of the image will be above, and in other parts below the mean value. After subtraction of the mean background value, regions with previously lower than average background will have negative pixel value.  This means that a forced photometry on a location where an object is undetected in single epoch may be measuring the background noise. Apart from creating low pixel values, background oscillation may also cause spurious detections where it is above the mean. For a large number of photons hitting the detector the width of the background noise distribution is  proportional to the square-root of counts: $\sigma_{B} \propto \sqrt {B}$. This yields a 1 in a million chance that a pixel has a background value larger than $5\sigma_{B}$.   Thus on a 16 megapixel CCD, like those used in SDSS, we anticipate  about  16 spuriously bright pixels  per CCD.  Therefore,  since forced photometry is affected by  the background noise, a special care must be taken of faint, noise-dominated measurements. 

\section{Assignment 5 paragraph}
\subsection{before} 
Forced photometry measurement in a single epoch  is in fact the mean of a likelihood for the observed flux. We can think of each source as  being represented by a Gaussian likelihood centered on the measured flux $F_{i}$, with a widh corresponding to the measurement error $\sigma_{i}$. Thus in the flux likelihood space bright sources have very narrow Gaussians, with a width on the level of $1-2 \%  \approx 0.01-0.02$ mag, whereas faint sources with larger uncertainties have very wide Gaussian tails, that can extend  below zero.  Such nonzero likelihood for a negative flux measurement is unphysical, given that we know that no source can in  reality have a negative flux. Any negative portion of the likelihood stems from the background fluctuation, as described in previous section, or from very large measurement error. We address the issue of negative tails of Gaussian flux likelihood by recalculating the measurement of flux for all sources  with less than $'2\sigma'$ detection, i.e. $ \langle F_{L} \rangle  < k \sigma$, with $k=2$. This corresponds to the $2\%$ probability of $F_{L} < 0$  (in a  Gaussian likelihood).
\subsection{after Ass 6 corrections}
We remedy the unphysically low, even negative measurements for all 'faint' ($< 2\sigma$) sources and recalculate their fluxes. Each flux measurement can  be thought of as a  mean of the 'intrinsic' flux likelihood function $L(F)$. $L$ determines the probability that the flux has a value $F$. In our treatment we assume that $L$ is a Gaussian, and therefore its width corresponds to the measurement error :  $L(F) \sim \mathcal{N}(F,\sigma_{F})$. This means that bright sources with high signal-to-noise ratio have very narrow  $L$, and faint sources, dominated by noise, have $L$ with very wide tails.  Only for faint sources a significant portion of $L$ may be negative, corresponding to non-zero likelihood of flux being negative. This stands in conflict with our prior knowledge that no physical flux can be negative. We resolve this problem by recalculating single-epoch flux for all sources where $F < 2 \sigma_{F}$. We calculate for each epoch the mean of the truncated $L$, such that $L(F) = 0$ for $F<0$.  This shifts upward all measurements for faint epochs, and remedies the unphysicality of faint forced photometry fluxes.   



\
%%%%%%%%%%%%%%%%%%%%%%%%%%%%%%%%%%%%%%%%%%%%%%%%%%

%%%%%%%%%%%%%%%%%%%% REFERENCES %%%%%%%%%%%%%%%%%%

% The best way to enter references is to use BibTeX:

%\bibliographystyle{mnras}
%\bibliography{example} % if your bibtex file is called example.bib

%%%%%%%%%%%%%%%%%%%%%%%%%%%%%%%%%%%%%%%%%%%%%%%%%%


% Don't change these lines
\bsp	% typesetting comment
\label{lastpage}
\end{document}

% End of mnras_template.tex
