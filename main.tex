% mnras_template.tex
%
% LaTeX template for creating an MNRAS paper
%
% v3.0 released 14 May 2015
% (version numbers match those of mnras.cls)
%
% Copyright (C) Royal Astronomical Society 2015
% Authors:
% Keith T. Smith (Royal Astronomical Society)

% Change log
%
% v3.0 May 2015
%    Renamed to match the new package name
%    Version number matches mnras.cls
%    A few minor tweaks to wording
% v1.0 September 2013
%    Beta testing only - never publicly released
%    First version: a simple (ish) template for creating an MNRAS paper

%%%%%%%%%%%%%%%%%%%%%%%%%%%%%%%%%%%%%%%%%%%%%%%%%%
% Basic setup. Most papers should leave these options alone.
\documentclass[fleqn,usenatbib]{mnras}  % a4paper,

% MNRAS is set in Times font. If you don't have this installed (most LaTeX
% installations will be fine) or prefer the old Computer Modern fonts, comment
% out the following line
%\usepackage{newtxtext,newtxmath}
\usepackage{txfonts}
% Depending on your LaTeX fonts installation, you might get better results with one of these:
%\usepackage{mathptmx}
%\usepackage{txfonts}


% Use vector fonts, so it zooms properly in on-screen viewing software
% Don't change these lines unless you know what you are doing
\usepackage[T1]{fontenc}
\usepackage{ae,aecompl}
\usepackage{slashbox}

%%%%% AUTHORS - PLACE YOUR OWN PACKAGES HERE %%%%%

% Only include extra packages if you really need them. Common packages are:
\usepackage{graphicx}	% Including figure files
\usepackage{amsmath}	% Advanced maths commands
\usepackage{amssymb}	% Extra maths symbols

%%%%%%%%%%%%%%%%%%%%%%%%%%%%%%%%%%%%%%%%%%%%%%%%%%

%%%%% AUTHORS - PLACE YOUR OWN COMMANDS HERE %%%%%

% Please keep new commands to a minimum, and use \newcommand not \def to avoid
% overwriting existing commands. Example:
%\newcommand{\pcm}{\,cm$^{-2}$}	% per cm-squared

%%%%%%%%%%%%%%%%%%%%%%%%%%%%%%%%%%%%%%%%%%%%%%%%%%

%%%%%%%%%%%%%%%%%%% TITLE PAGE %%%%%%%%%%%%%%%%%%%

% Title of the paper, and the short title which is used in the headers.
% Keep the title short and informative.
\title[SDSS Quasars]{SDSS Stripe 82 : quasar variability from forced photometry}

% The list of authors, and the short list which is used in the headers.
% If you need two or more lines of authors, add an extra line using \newauthor
\author[K. Suberlak et al.]{
Krzysztof Suberlak,$^{1}$\thanks{E-mail: suberlak@uw.edu}
\v{Z}eljko Ivezi\'c, $^{1}$
Yusra AlSayyad,$^{1}$ 
\\
% List of institutions
$^{1}$Department of Astronomy, University of Washington, Seattle, WA, United States\\
}

% These dates will be filled out by the publisher
\date{Accepted XXX. Received YYY; in original form ZZZ}

% Enter the current year, for the copyright statements etc.
\pubyear{2015}

% Don't change these lines
\begin{document}
\label{firstpage}
\pagerange{\pageref{firstpage}--\pageref{lastpage}}
\maketitle

% Abstract of the paper
\begin{abstract}

\end{abstract}


%%%%%%%%%%%%%%%%%%%%%%%%%%%%%%%%%%%%%%%%%%%%%%%%%%

%%%%%%%%%%%%%%%%% BODY OF PAPER %%%%%%%%%%%%%%%%%%

\section{Introduction}
\label{sec:intro}


\section{Data Analysis}
\label{sec:data}

We use data from all SDSS runs up to an including run 7202 (Data Release 7), including all 6 SDSS camera columns. 

All epochs (individual images) were background-subtracted, and then scaled from the Digital Unit counts to fluxes by comparing standard objects against the Ivezic+2007 catalog  (similar to  Jiang+2014).   

The data were coadded, and all objects detected in the i-band coadds were assigned a deepSourceId (== objectId). For star-galaxy separation, the entire clump was considered as one parent source (with single ParentSourceId). For an object which is a parent (eg. a galaxy), ParentSourceId is null.  This amounts to $~40$ milion  i-band detections down to $3 \, \sigma$.  $8$ million of those are brighter than $23^{rd}$ mag. Thus the total number of photometric measurements is : ($40$  million i-band detections) x ($80$ epochs ) x ($5$ filters) = $~16$ billion measurements, including   ($8$ million i-band detections i < 23) x ($80$ epochs) x ($5$ filters ) = $~3$ billion measurements brighter than $23^{rd}$ mag.

Forced photometry was performed in u,g,r,i,z  on the individual epoch images (NOT difference imaging),  in locations specified by i-band coadds. (DIFFERENCE imaging is when photometry is done on  coadd - individual-epoch-image) . Therefore in some cases the flux reported for a given aperture is negative, because after background subtraction noise oscillates around 0, and when it is scaled up,  it can have negative values. [stored in rawDataFPSplit ]  

The average brightness of an object in  a given filter can be found in two ways. The median of the forced photometry values (over all epochs, including the negative fluxes)   will better reflect the actual brightness of a variable source. This may mean that the median is negative, i.e. the median flux is negative. Since the magnitude is undefined for negative fluxes, we then revert to the lightcurve, and for each exposure with negative flux we find the zero point magnitude ($m1$) - the magnitude for a source with a flux of 1 count per sec, different for each exposure.  The zero point magnitude for each exposure with negative flux is calculated from the  Flux of 0 magnitude source,  $F0$,  as  $m1 = 2.5 \log_{10}{F0}$. For that object the new median magnitude in that filter will be the upper limit. 

Colors can be calculated in two ways: using the median of forced photometry over all epochs (object detected in coadded i-band has photometry in all epochs), or the median over single-epoch detections (only when an object was above the detection threshold for a single epoch).  
Thus the median over all detections will be biased (especially for faint sources) towards higher brightness.  On the other hand, the median over all epochs will be more representative of the true brightness of an object in a given filter.  If a median brightness is negative, we use zero point magnitudes and in such cases median over all epochs will be an upper limit on brightness, but still less biased than median over all detections. Therefore  we choose to use median over all epochs to calculate colors.  

Since the reported fluxes are not extinction-corrected, we use a table of E(B-V) in a direction of a given source to correct for the galactic extinction. We use the formula  $x_{corr}  = x_{obs} + A_{x} * E(B-V)$, where $x$ is  u,g,r,i,z , and $A_x$ is 5.155, 3.793, 2.751, 2.086, 1.479  for each filter respectively  [SOURCE? ] 



\section{Results}
\label{sec:results}

\section{Conclusions}
\label{sec:conclusions}



\section*{Acknowledgements}

Funding for the SDSS and SDSS-II has been provided by the Alfred P. Sloan Foundation, the Participating Institutions, the National Science Foundation, the U.S. Department of Energy, the National Aeronautics and Space Administration, the Japanese Monbukagakusho, the Max Planck Society, and the Higher Education Funding Council for England. The SDSS Web Site is http://www.sdss.org/.

The SDSS is managed by the Astrophysical Research Consortium for the Participating Institutions. The Participating Institutions are the American Museum of Natural History, Astrophysical Institute Potsdam, University of Basel, University of Cambridge, Case Western Reserve University, University of Chicago, Drexel University, Fermilab, the Institute for Advanced Study, the Japan Participation Group, Johns Hopkins University, the Joint Institute for Nuclear Astrophysics, the Kavli Institute for Particle Astrophysics and Cosmology, the Korean Scientist Group, the Chinese Academy of Sciences (LAMOST), Los Alamos National Laboratory, the Max-Planck-Institute for Astronomy (MPIA), the Max-Planck-Institute for Astrophysics (MPA), New Mexico State University, Ohio State University, University of Pittsburgh, University of Portsmouth, Princeton University, the United States Naval Observatory, and the University of Washington. 


%%%%%%%%%%%%%%%%%%%%%%%%%%%%%%%%%%%%%%%%%%%%%%%%%%

%%%%%%%%%%%%%%%%%%%% REFERENCES %%%%%%%%%%%%%%%%%%

% The best way to enter references is to use BibTeX:

%\bibliographystyle{mnras}
%\bibliography{example} % if your bibtex file is called example.bib

%%%%%%%%%%%%%%%%%%%%%%%%%%%%%%%%%%%%%%%%%%%%%%%%%%


% Don't change these lines
\bsp	% typesetting comment
\label{lastpage}
\end{document}

% End of mnras_template.tex
